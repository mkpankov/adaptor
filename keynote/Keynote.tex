\documentclass{beamer}
\usepackage[T2A]{fontenc}
\usepackage[english,russian]{babel}
\usepackage[utf8x]{inputenc}

\usetheme{Copenhagen}

\title{Инструментарий моделирования быстродействия программ}
\author{Михаил Панков}
\institute{МГТУ им Н.Э.Баумана}
\date{19 июня 2013 года}

\makeatletter
%\definecolor{beamer@blendedblue}{rgb}{0.5,0.5,0.3} % changed this
\definecolor{beamer@darkgreen}{RGB}{35,150,35}

%\setbeamercolor{normal text}{fg=black,bg=white}
%\setbeamercolor{alerted text}{fg=red}
%\setbeamercolor{example text}{fg=green!50!black}

\setbeamercolor{structure}{fg=beamer@darkgreen}

%\setbeamercolor{background canvas}{parent=normal text}
%\setbeamercolor{background}{parent=background canvas}

\setbeamercolor{palette primary}{bg=beamer@darkgreen} % changed this
\setbeamercolor{palette secondary}{use=structure,fg=structure.fg!100!green} % changed this
\setbeamercolor{palette tertiary}{use=structure,fg=structure.fg!100!green} % changed this
\makeatother

\begin{document}

\maketitle

\begin{frame}
\frametitle{Цель, задачи, и актуальность темы работы}

\begin{block}{Цель}
	\begin{itemize}
		\item Разработать метод и ПО (инструментарий) моделирования быстродействия программ на компьютерах общего назначения
	\end{itemize}
\end{block}

\begin{block}{Задачи}
	\begin{enumerate}
		\item Разработать метод моделирования быстродействия программ
		\item Выполнить программную реализацию инструментария, позволяющего оценивать быстродействие программ и осуществлять его моделирование
		\item Исследовать эффективность инструментария моделирования быстродействия программ на тестовом наборе программ
	\end{enumerate}
\end{block}

\begin{block}{Актуальность}
	\begin{enumerate}
		\item Оценка эффективности компьютера на этапе его проектирования или приобретения
		\item Поиск оптимальных настроек компилятора при итеративной
	\end{enumerate}
\end{block}

\end{frame}

\end{document}