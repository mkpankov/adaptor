\section{Охрана труда и экология}
\subsection{Проектирование рабочего места оператора ПЭВМ}
Требования к компьютерной технике и к условиям работы с ней в Российской Федерации регламентируются санитарными нормами и правилами СанПиН 2.2.2/2.4.1340-03 и СанПиН 2.2.2.000-02. Рассмотрим основные нормы, необходимые для проектирования рабочего места оператора ПЭВМ.

\subsubsection{Требования к рабочим помещениям}
Согласно СанПиН 2.2.2/2.4.1340-03, помещения для работы с компьютерами должны оборудоваться системами отопления, кондиционирования воздуха или эффективной приточно-вытяжной вентиляцией. Звукоизоляция помещений и звукопоглощение ограждающих конструкций помещения должны отвечать гигиеническим требованиям и обеспечивать нормируемые параметры шума на рабочих местах. Помещения должны иметь естественное и искусственное освещение.

Поверхность пола в помещениях должна быть ровной, без выбоин, нескользкой, удобной для очистки и влажной уборки, обладать антистатическими свойствами. При строительстве новых и реконструкции действующих средних, средних специальных и высших учебных заведений помещения для работы с компьютером следует проектировать высотой (от пола до потолка) не менее 4,0 м.

Расположение рабочих мест для взрослых пользователей в подвальных помещениях не допускается.

\subsubsection{Требования к освещению}
Естественное освещение должно осуществляться через светопроемы, ориентированные преимущественно на север и северо-восток. Оконные проемы в помещениях использования компьютеров должны быть оборудованы регулируемыми устройствами типа жалюзи, занавесей, внешних козырьков и др. Система естественного освещения должна обеспечивать коэффициент естественной освещенности (КЕО) не ниже 1,5.

Искусственное освещение в помещениях должно осуществляться системой общего равномерного освещения. В производственных и административно - общественных помещениях, в случаях преимущественной работы с документами, допускается применение системы комбинированного освещения (к общему освещению дополнительно устанавливаются светильники местного освещения, предназначенные для освещения зоны расположения документов). 

Освещенность на поверхности стола в зоне размещения рабочего документа должна быть 300-500~лк. Допускается установка светильников местного освещения для подсветки документов. Местное освещение не должно создавать бликов на поверхности экрана и увеличивать освещенность экрана более 300~лк. 

Следует ограничивать прямую блескость от источников освещения, при этом яркость светящихся поверхностей (окна, светильники и др.), находящихся в поле зрения, должна быть не более 200~кд/кв. м. В качестве источников света при искусственном освещении должны применяться преимущественно люминесцентные лампы типа ЛБ. При устройстве отраженного освещения в производственных и административно-общественных помещениях допускается применение металлогалогенных ламп мощностью до 250 Вт. Допускается применение ламп накаливания в светильниках местного освещения. 

Для освещения помещений следует применять светильники серии ЛПО36 с зеркализованными решетками, укомплектованные высокочастотными пускорегулирующими аппаратами (ВЧ ПРА). Коэффициент пульсации не должен превышать 5\%, что должно обеспечиваться применением газоразрядных ламп в светильниках общего и местного освещения с ВЧ ПРА. 

Допускается применять светильники серии ЛПО36 без ВЧ ПРА только в модификации «Кососвет», а также светильники прямого света -- П, преимущественно прямого света -- Н, преимущественно отраженного света -- В. При этом лампы многоламповых светильников или рядом расположенные светильники общего освещения следует включать на разные фазы трехфазной сети.

Применение светильников без рассеивателей и экранирующих решеток не допускается. Яркость светильников общего освещения в зоне углов излучения от 50\textdegree до 90\textdegree с вертикалью в продольной и поперечной плоскостях должна составлять не более 200 кд/кв. м, защитный угол светильников должен быть не менее 40\textdegree. Светильники местного освещения должны иметь непросвечивающий отражатель с защитным углом не менее 40\textdegree.

Для обеспечения нормируемых значений освещенности в помещениях использования ВДТ и ПЭВМ следует проводить чистку стекол оконных рам и светильников не реже двух раз в год и проводить своевременную замену перегоревших ламп.

Коэффициент запаса (Кз) для осветительных установок общего освещения должен приниматься равным 1,4. 

Для внутренней отделки интерьера помещений должны использоваться диффузно-отражающие материалы с коэффициентом отражения 
\begin{itemize}
	\item для потолка $\rho_{\pe\co\te} = 0,7-0,8$;
	\item для стен $\rho_{\es\te} = 0,5-0,6$;
	\item для пола $\rho_{\pe\co\el} = 0,3-0,5$;
\end{itemize}

\subsubsection{Расчёт системы освещения в помещении}

В помещении, где находится рабочее место оператора, используется смешанное освещение,  т.е. сочетание естественного и искусственного освещения. В качестве естественного - боковое освещение через окно. Искусственное освещение используется при недостаточном естественном освещении. В данном помещении используется общее искусственное освещение.

Расчет его осуществляется по методу светового потока с учетом потока, отраженного от стен и потолка.

Нормами для данных работ установлена необходимая освещенность рабочего места $E_{\en}=300\el\ka$ (средняя точность работы по различению деталей размером от 1 до 10 мм). 

Площадь помещения:
\begin{equation*}
	S = A \cdot B = 4 \cdot 6 = 24 \cm.
\end{equation*}

Общий световой поток определяется по формуле:
\begin{equation*}
	F = \frac{E \cdot K \cdot S \cdot Z}{n},
\end{equation*}
где $E$ -- нормированная минимальная освещённость, лк. Работа программиста относится к разряду точных работ, следовательно, минимально освещённость будет $E = 300 \el\ka$ при газоразрядных лампах;

$Z$ -- отношение средней освещённости к минимальной (обычно принимается равным $1,1-1,2$, положим $Z = 1,1$);

$K$ -- коэффициент запаса, учитывающий уменьшение светового потока лампы в результате загрязнения светильников в процессе эксплуатации (его значение определяется по таблице коэффициентов запаса для различных помещений и в нашем случае $К = 1,5$);

$n$ -- коэффициент использования (выражается отношением светового потока, падающего на расчетную поверхность, к суммарному потоку всех ламп и исчисляется в долях единицы; зависит от характеристик светильника, размеров помещения, окраски стен и потолка, характеризуемых коэффициентами отражения от стен $P_c$ и потолка $P_{\pe}$). Значение коэффициентов $P_c$ и $P_{\pe}$ определим по таблице зависимостей коэффициентов отражения от характера поверхности: $P_c=30\%$, $P_{\pe}=50\%$. Значение $n$ определим по таблице коэффициентов использования различных светильников. Для этого вычислим индекс помещения по формуле:
\begin{equation*}
	I = \frac{S}{h \cdot (A + B)},
\end{equation*}
где $S$ -- площадь помещения, $S = 24 \cm^2$;

$h$ -- расчётная высота подвеса, $h = 3 \cm$;

$A$ -- длина помещения, $A = 6 \cm$;

$B$ -- ширина помещения, $B = 4 \cm$.

Подставив значения, получим:

\begin{equation*}
	I = \frac{24}{3 \cdot (6 + 4)} = 0,8
\end{equation*}

Зная индекс $i$, $P_c$, $P_{\pe}$, по таблице находим, что значения $n = 0,36$.

Подставим все значения в формулу для определения светового потока $F$:

\begin{equation*}
	F = \frac{300 \cdot 1,5 \cdot 24 \cdot 1,1}{0,36} = 33000 \el\cm
\end{equation*}

Люминесцентные лампы имеют ряд преимуществ перед лампами накаливания: их спектр ближе к естественному, обладают более высоким КПД (в 1,5-2 раза выше, чем КПД ламп накаливания), имеют большую экономичность (больше светоотдача) и срок службы (в 10-12 раз). Наряду с этим имеются и недостатки: их работа сопровождается иногда шумом; хуже работают при низких температурах; их нельзя применять во взрывоопасных помещениях; имеют малую инерционность. Для нашего помещения люминесцентные лампы подходят.

Для освещения выбираем люминесцентные лампы типа ЛБ40-1, световой поток которых F = 4320 лм. 

Рассчитаем необходимое количество ламп по формуле:

\begin{equation*}
	N = \frac{F}{F_{\EL}},
\end{equation*}
где $N$ -- определяемое число ламп;

$F$ -- световой поток, $F = 33000 \el\cm$;

$F_{\el}$ -- световой поток лампы, $F_{\el} = 4320 \el\cm$.

\begin{equation*}
	N = \frac{33000}{4320} = 8 \csh\te.
\end{equation*}

При выборе осветительных приборов используем светильники типа ОД. Каждый светильник комплектуется двумя лампами ЛБ40. Размещаются светильники двумя рядами, по два в каждом ряду.

Рассчитаем суммарную мощность осветительной установки общего назначения:

\begin{equation*}
	P \sum = 40 \cdot 16 = 640 \VE\te.
\end{equation*}

\subsubsection{Требования к микроклимату}

Оптимальным температурным режимом работы принимается температура окружающей среды от 19\textdegree до 21\textdegree и соответствующие им значения относительной влажности от 62\% до 55\%. Абсолютная влажность воздуха должна быть около 10г/$\cm^3$. Скорость движения воздуха требуется ограничить значением менее 0,1 м/с . Для наиболее производительной и успешной работы инженера-программиста (пользователя ПЭВМ), требуется обеспечить характеристики микроклимата по значениям, не уступающим вышеизложенным.

Вышеуказанные требования СанПиН были соблюдены на рабочем месте, где проходило выполнение данного дипломного проекта. Система отопления включала основную и вспомогательную системы. Основная система – система приточной вентиляции с подогревом воздуха, также выполняющая роль системы вентиляции помещения. Вспомогательная – статический обогрев помещения за счет батарей центрального отопления. Контроль за соблюдением техники безопасности и удовлетворением норм микроклимата регулярно проводится инженером по технике безопасности. Рабочее место аттестовано сертифицированными службами и допущено к эксплуатации.

\subsection{Расчёт системы вентиляции}
Помещения, в которых производится эксплуатации ПЭВМ, должны быть оборудованы системами приточно-вытяжной вентиляции, при работе которой будут обеспечены допустимые параметры микроклимата в помещении. Расчет производят только для вытяжной ветви приточно-вытяжной системы вентиляции.

$V_{\ve\e\en\te}$ -- объем воздуха, необходимый для обмена, $\cm^3$/ч;
$V_{\pe\co\cm}$ -- объем рабочего помещения, $\cm^3$.
Для расчета примем следующие размеры рабочего помещения: 
\begin{itemize}
	\item длина $В = 6 \cm$; 
	\item ширина $А = 4 \cm$; 
	\item высота $Н = 3 \cm$. 
\end{itemize}

Соответственно, объем помещения равен: 
\begin{equation*}
	V_{\pe\co\cm} = A \cdot B \cdot H = 72 \cm^3.
\end{equation*}

Необходимый для обмена объём воздуха $V_{\ve\e\en\te}$ определим исходя из уравнения теплового баланса:
\begin{equation*}
	V_{\ve\e\en\te} \cdot (t_{\cu\ha\co\de} - t_{\pe\re\ci\ha\co\de}) \cdot Y = 3600 \cdot Q_{\ci\ze\be},
\end{equation*}
где $Q_{\ci\ze\be}$ -- избыточная теплота (Вт);

$C = 1005 \DE\zhe / (\ka\ge \cdot \KA)$ -- удельная теплопроводная воздуха;

$Y = 1,2 \cm\ge / \es\cm^3$ -- плотность воздуха.

Температура уходящего воздуха определяется по формуле:
\begin{equation*}
	t_{\cu\ha\co\de} = t_{\re\cm} + (H - 2) \cdot t,
\end{equation*}
где $t = 1-5^\circ$ -- превышение t на 1м высоты помещения;

$t_{\re\cm} = 21^\circ$ -- температура на рабочем месте;

H = $3 \cm$ -- высота помещения;

\begin{eqnarray*}
	t_{\cu\ha\co\de} = 21 + (3 - 2) \cdot 4,7 = 25,47^\circ \\
	Q_{\ci\ze\be} = Q_{\ci\ze\be1} + Q_{\ci\ze\be2} + Q_{\ci\ze\be3},
\end{eqnarray*}
где $Q_{\ci\ze\be}$ -- избыток тепла от электрооборудования и освещения.

\begin{equation*}
	Q_{\ci\ze\be1} = E \cdot p
\end{equation*}
где $E$ -- коэффициент потерь электроэнергии на теплоотвод ($E = 0,55$ для освещения);

$p$ -- мощность, $p = 40 \VE\te \cdot 8 + 500 \VE\te = 820\VE\te$.
\begin{equation*}
	Q_{\ci\ze\be1} = 0,55 \cdot 820 = 461 \VE\te,
\end{equation*}
$Q_{\ci\ze\be2}$ -- теплопоступление от солнечной радиации,
\begin{equation*}
	Q_{\ci\ze\be2} = m \cdot S \cdot k \cdot Q_c,
\end{equation*}
где $m$ -- число окон, примем $m = 2$;

$S$ -- площадь окна, $S = 2,2 \cdot 1,5 = 3,3 \cm^2$;

$k$ -- коэффициент, учитывающий остекление (для двойного остекления $k = 0,6$);

$Q_c = 127 \VE\te / \cm$ -- теплопоступление от окон.

\begin{equation*}
	Q_{\ci\ze\be2} = 3,3 \cdot 2 \cdot 0,6 \cdot 127 = 505,92 \VE\te.
\end{equation*}

$Q_{\ci\ze\be3}$ -- тепловыделение людей.

\begin{equation*}
	Q_{\ci\ze\be3} = n \cdot q,
\end{equation*}
где $q = 80 \VE\te / \che\e\el.$,

$n$ -- число людей ($n = 1$).

Тогда

\begin{eqnarray*}
	Q_{\ci\ze\be3} = 1 \cdot 80 = 80 \VE\te \\
	Q_{\ci\ze\be} = 461 + 502 + 80 = 1143 \VE\te
\end{eqnarray*}

Из уравнения теплового баланса следует:

\begin{equation*}
	V_{\ve\e\en\te} = \frac{3600 \cdot 1143}{1005 \cdot 1,2 \cdot (25,47 - 19)} = 527,35 \cm^3 / \che\ca\es
\end{equation*}

\subsubsection{Выбор вентилятора}
В нашем случае будет использоваться приточно-вытяжная вентиляция.

Вентиляционная система состоит из следующих элементов (рисунок~\ref{img:ventilation}): 
\begin{enumerate}
	\item забор воздуха;
	\item система кондиционирования;
	\item вентилятор;
	\item решетка;
	\item стальной воздуховод с круглым сечением;
	\item стальной воздуховод с круглым сечением;
	\item выброс воздуха.
\end{enumerate}

\begin{figure}[H]
    \center{\includegraphics[width=0.5\linewidth]{ventilation}}
    \caption{Вентиляционная система.}
    \label{img:ventilation}
\end{figure}

Исходными данными для выбора вентилятора является расчётная производительность вентилятора:

\begin{equation*}
	V_{\re\ca\es\che} = 1,1 \cdot V_{\ve\e\en\te} = 1,1 \cdot 527 = 579,7 \cm^3 / \che\ca\es,
\end{equation*}
где $1,1$ -- коэффициент, учитывающий утечки и подсосы воздуха.

Потери давления в вентиляционной системе определяются по формуле:
\begin{equation*}
	H = R \cdot l + \xi \cdot \frac{V^2 \cdot p}{2},
\end{equation*}
где $H$ -- потери давления, Па;

$R$ -- увдельные потери давления на трение в воздуховоде, Па/м;

$I = 4,5 \cm$ -- длина воздуховода;

$V = 7 \cm / \es$ -- скорость воздуха;

$p = 1,2 \ka\ge / \cm$ -- плотность воздуха.

Необходимый диаметр воздуховода для данной вентиляционной системы:

\begin{equation*}
	L = \frac{d^2 \cdot \pi}{4} \cdot W,
\end{equation*}
где $L = V_{\re\ca\es\che} = 579,7 \cm^3 / \che\ca\es$;

$W = 7 \cm / \es$.

\begin{equation*}
	d = \sqrt{ \frac{4 \cdot L}{\pi \cdot W}} = 0,17 \cm.
\end{equation*}

Принимаем в качестве диаметра ближайшую большую стандартную величину -- 0,2 м, при которой удельные потери давления на трение в воздуховоде -- $R = 3,58 \PE\ca / \cm$.

После перехода на стандартный диаметр производим перерасчёт скорости:

\begin{equation*}
	W_{\en\co\re} = \frac{4}{d^2 \cdot \pi} \cdot L = 5,12 \cm / \es.
\end{equation*}

Местные потери возникают в железной решётке ($\xi = 1,2$), в изгибе трубопровода ($\xi = 1,7$). Отсюда суммарный коэффициент местных потерь в системе:

\begin{equation*}
	\xi = (1,2 \cdot 3) + 1,7 = 5,3.
\end{equation*}

Тогда

\begin{equation*}
	 H = 3,58 \cdot 4,5 + 5,3 \cdot \frac {26,2 \cdot 1,2}{2} = 99,4 \PE\ca.
\end{equation*}

С учётом 10\% запаса:

\begin{equation*}
	H = 1,1 \cdot 99,4 = 109,3 \PE\ca.
\end{equation*}

Далее в проводимом расчёте по вычисленному напору определяется марка вентилятора из соответствующих таблице. Рассмотрим, например, соответствующую таблицу для вентиляторов радиальных (центробежных) серии ВЦ-4-70 (таблица~\ref{tab:fans}).

\begin{table}[H]
    \caption{\label{tab:fans} Вентиляторы радиальные (центробежные) серии ВЦ-4-70.}
    \begin{center}
        \begin{tabular}{|l|c|c|c|c|c|c|c|c|}
            \hline
             &
            \multicolumn{2}{|c|}{Электродвигатель} &
            \multicolumn{2}{|c|}{\specialcell{Характеристики\\вентилятора}} &
            \multicolumn{4}{|c|}{Исполнение}\\
            \cline{2-9}
            \raisebox{-1ex}[5cm][0cm]{\specialcell{Марка\\вент.}} &
            \specialcell{Мощность,\\кВт} &
            \specialcell{Частота\\вращения,\\об./мин.} &
            \specialcell{Расход\\воздуха,\\$\cm^3 \cdot 10^3$/час} &
            \specialcell{Напор,\\Па} &
            \rotatebox{90}{\specialcell{Обычное}} &
            \rotatebox{90}{\specialcell{Коррозионноестойкое}} &
            \rotatebox{90}{\specialcell{Искрозащищённое}} &
            \rotatebox{90}{\specialcell{Взрывозащищённое}} \\
            \hline
            \multirow{2}{*}{\specialcell{ЦВ-4-\\70-2,5}} &
            0,18 & 1500 & 0,37-0,92 & 140-190 & + & + & + & + \\
             & 0,75 & 3000 & 0,75-1,92 & 500-760 & + & + & + & + \\
            \hline
            \multirow{2}{*}{\specialcell{ЦВ-4-\\70-3,15}} &
            0,27 & 1500 & 0,78-2,0 & 190-310 & + & + & + & + \\
             & 1,5 & 3000 & 1,65-4,10 & 800-1300 & + & + & + & + \\
            \hline
            \multirow{3}{*}{\specialcell{ЦВ-4-\\70-4}} &
            0,18 & 950 & 1,15-2,6 & 130-220 & + & + & + & + \\
             & 1,1 & 1500 & 1,85-4,0 & 300-490 & + & + & + & + \\
             & 7,5 & 3000 & 3,6-7,2 & 1200-2250 & + & + & + & + \\
            \hline
            \multirow{2}{*}{\specialcell{ЦВ-4-\\70-5}} &
            0,55 & 950 & 2,3-5,1 & 160-360 & + & + & + & + \\
             & 2,2 & 1500 & 3,5-8,0 & 380-840 & + & + & + & + \\
            \hline
            \multirow{2}{*}{\specialcell{ЦВ-4-\\70-6,3}} &
            1,5 & 950 & 4,2-10,2 & 310-500 & + & + & + & + \\
             & 5,5 & 1500 & 7,0-16,0 & 580-1280 & + & + & + & + \\
            \hline
        \end{tabular}
    \end{center}
\end{table}

По этим данным выбираем вентилятор ВЦ-4-70-2,5:
\begin{itemize}
	\item расход воздуха: $370-920 \cm^3 / \che\ca\es$;
	\item давление: 140-190 Па;
	\item скорость вращения: 1500 об./мин.;
	\item мощность электродвигателя: 0,18 кВт.
\end{itemize}