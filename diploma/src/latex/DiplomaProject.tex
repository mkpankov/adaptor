% Размер бумаги А4, шрифт 12 пунктов
\documentclass[a4paper,12pt]{report}
% Перекодировка шрифтов
\usepackage[T2A]{fontenc}
% Кодировка: utf8
\usepackage[utf8]{inputenc}
% Используем русский и английский языки с переносами
\usepackage[english,russian]{babel}
% Подключаем нужные пакеты расширений
\usepackage{amssymb,amsfonts,amsmath,mathtext,cite,enumerate,float}

% Вставка рисунков
\usepackage{graphicx}
% Путь к рисункам
\graphicspath{{../pictures/}}

% Подсчёт объектов
\usepackage{totcount}
\regtotcounter{page}
\regtotcounter{figure}
\regtotcounter{table}

% Изменение размеров шрифтов
\usepackage{scrextend}
% Размер по умолчанию — 14 пунктов, интервал полуторный (14*1.25)
\changefontsizes[17.5pt]{14pt}
% Отступ первой строки параграфа после начала раздела
\usepackage{indentfirst}
% Отступ начала параграфа
\setlength{\parindent}{1cm}

% Отступы слева для списков
% Первый уровень выровнен на отступ нового абзаца, 
% остальные -- на начало элемента более высокого уровня.
\usepackage{enumitem}
\setlist[itemize,1]{leftmargin=1.55cm, label=---}
\setlist[enumerate,1]{leftmargin=1.55cm}
\setlist[itemize,2,3]{leftmargin=*, label=---}
\setlist[enumerate,2,3]{leftmargin=*}

% Заменяем библиографию с квадратных скобок на точку
\makeatletter
\renewcommand{\@biblabel}[1]{#1.}
\makeatother

% Геометрия страницы
\usepackage{geometry}
% Поля
\geometry{left=3cm}
\geometry{right=1cm}
\geometry{top=2cm}
\geometry{bottom=2cm}

% Меняем везде перечисления на цифра.цифра
\renewcommand{\theenumi}{\arabic{enumi})}
\renewcommand{\labelenumi}{\arabic{enumi})}
\renewcommand{\theenumii}{\arabic{enumii})}
\renewcommand{\labelenumii}{\arabic{enumi}.\arabic{enumii})}
\renewcommand{\theenumiii}{\arabic{enumiii})}
\renewcommand{\labelenumiii}{\arabic{enumi}.\arabic{enumii}.\arabic{enumiii})}

% Изменяем \part и \chapter из report.cls чтобы они не выводили нумерованные заголовки
% Рецепт отсюда: http://tex.stackexchange.com/questions/70621/part-and-chapter-commands-output-both-numbered-and-named-headings/70624
\makeatletter
    \def\@part[#1]#2{%
        \ifnum \c@secnumdepth >-2\relax
        \refstepcounter{part}%
        \addcontentsline{toc}{part}{#1}%
        \else
        \addcontentsline{toc}{part}{#1}%
        \fi
        \markboth{}{}%
        {
            \centering
            \interlinepenalty \@M
            \normalfont
            \Huge \bfseries #2\par}%
        \@endpart}
    \def\@chapter[#1]#2{\ifnum \c@secnumdepth >\m@ne
        \refstepcounter{chapter}%
        \typeout{\@chapapp\space\thechapter.}%
        \addcontentsline{toc}{chapter}%
        {#1}%
        \else
        \addcontentsline{toc}{chapter}{#1}%
        \fi
        \chaptermark{#1}%
        \addtocontents{lof}{\protect\addvspace{10\p@}}%
        \addtocontents{lot}{\protect\addvspace{10\p@}}%
        \if@twocolumn
        \@topnewpage[\@makechapterhead{#2}]%
        \else
        \@makechapterhead{#2}%
        \@afterheading
        \fi}
    \def\@makechapterhead#1{%
        \vspace*{50\p@}%
        {\parindent \z@ \raggedright \normalfont
            \interlinepenalty\@M
            \Huge \bfseries #1\par\nobreak
            \vskip 40\p@
    }}
    \renewcommand{\thesection}{\arabic{section}} % Удаляем \thechapter из \thesection
\makeatother

% Нумерация подразделов
\setcounter{secnumdepth}{3}
% Включение подразделов в оглавление
\setcounter{tocdepth}{3}
% Для веб-адресов в библиографии
\usepackage{url}
% Для подчёркиваний в полях библиографических записей
\usepackage[strings]{underscore}
% Библиография в алфавитном порядке с цифровой нумерацией
\bibliographystyle{ugost2008}
% Определяем названия месяцев вручную
\def\bbljan{Январь}
\def\bblfeb{Февраль}
\def\bblmar{Март}
\def\bblapr{Апрель}
\def\bblmay{Май}
\def\bbljun{Июнь}
\def\bbljul{Июль}
\def\bblaug{Август}
\def\bblsep{Сентябрь}
\def\bbloct{Октябрь}
\def\bblnov{Ноябрь}
\def\bbldec{Декабрь}

% Выключаем переносы
\makeatletter\chardef\l@nohyphenation=255 \makeatother
\usepackage[english=nohyphenation,russian=nohyphenation]{hyphsubst}
% Не хотим строки, которые влезают на поля
\sloppy

% !TeX spellcheck = ru_RU
\begin{document}
% Титульный лист
\input{PracticeReport-Title}
% Учебное задание
\setcounter{page}{2}
\section*{Учебное задание}
Произвести анализ производительности программ с помощью инструментария, разработанного в рамках курсового проекта. В рамках выполнения этого задания произвести шаги, перечисленные ниже.
\begin{itemize}
	\item Построить первоначальную образцовую модель производительности.
	\item Выявить недостатки первоначальной модели.
	\item Разработать программу анализа модели производительности с целью выявления недостаточности имеющихся в модели признаков.
	\item Произвести добавление в модель признаков, позволяющих более точно описать производительность программы на данной платформе, в ручном или автоматическом режиме.
	\item Произвести оценку разработанной программы на наборе моделей, полученных с помощью анализа программ, входящих в состав набора Polybench.
\end{itemize}

\pagebreak
% Содержание
\tableofcontents
\pagebreak
% Введение
\section{Введение}
Разработка современного оптимизирующего промышленного компилятора --- сложная задача, требующая огромных вложений человеческих ресурсов. Больш\'{а}я часть этой сложности обусловлена необходимостью тонкой настройки эвристических методов оптимизации исполняемого кода, производимой вручную путём просмотра генерируемого кода и дампов промежуточного представления структур компилятора. Более того, сами правила обнаружения случаев, в которых можно оптимизировать код, создаются человеком, что также трудоёмко и является далеко не исчерпывающим способом увеличения производительности результирующего кода. Принципиально задача подбора самих эвристик и их настройки может решаться автоматически, как уже доказали работы \cite{Agakov:2006:UML:1121992.1122412,Bodin98iterativecompilation,FCA2007,Cooper:2005:AAC:1065910.1065921}.

В существующих компиляторах также стоит проблема подбора оптимальных настроек компиляции для данной конкретной программы. Чаще всего проще ограничиться стандартной настройкой «максимальная производительность», нежели производить трудоёмкий анализ кода, который так же требует квалификации опытного разработчика компилятора и знания о платформе, для которой генерируется исполняемый файл. Эта задача также может решаться автоматически и, в ряде случаев, более эффективно.

Одним из способов решения этой задачи является итеративная компиляция. Она способна обеспечить компилятор возможностью быстро подстраиваться под сложные архитектуры и повысить его производительность относительно классических статических компиляторов — т.е. таких, которые не были специально модифицированы для работы с инструментарием машинного обучения \cite{Dubach:2009:PCO:1669112.1669124,Dubach:2008:EPA:1450095.1450103}.

Основной проблемой итеративной компиляции является большое количество необходимых для обучения запусков компилятора, причём все компиляции одной программы должны производиться с одинаковыми наборами данных на одной и той же архитектуре с помощью одного и того же (вплоть до версии) компилятора. Количество необходимых компиляций составляет десятки и сотни запусков.

В то же время, предпочтительным является обучение с использованием только производственных запусков компилятора, т.е. без специальных обучающих запусков. Для этого предлагается использовать коллективную оптимизацию, позволяющую пользователям обмениваться данными об успешной оптимизации и не проделывать избыточную работу каждый раз изолированно.

Перед реализацией коллективной оптимизации стоит несколько проблем исследовательского и инженерного характера. Инженерная проблема состоит в создании системы, позволяющей прозрачно переиспользовать накопленные другими пользователями знаний о произведенных компиляциях и отправлять данные о них без дополнительных действий со стороны конечного пользователя компилятора. Исследовательская проблема состоит в изучении воздействия различных оптимизаций на различные факторы получаемых программ, такие, как время исполнения и объём используемой памяти.

Коллективная оптимизация может обеспечить эффективность итеративной компиляции без произведения такого большого количества избыточных запусков компилятора. Применяемый подход предполагает статистическое сравнение эффективности пар наборов оптимизаций. В ходе описываемой работы также была обнаружена значительно более высокая важность обучения на различных наборах данных и на различных архитектурах, нежели обучения на различных программах. Это опровергает доминировавшую до этого точку зрения на применение машинного обучения в компиляторах.

Помимо рассмотрения результатов исследований воздействия оптимизаций на программы, описывается решение инженерной задачи: плагин к компилятору GCC, обеспечивающий его взаимодействие с центральным репозиторием результатов обучения.
% Обзор области
\input{DiplomaProject-Overview}
% Основная часть
% Мотивация, задачи, план
\chapter{Основная часть}
\section{Мотивация}
\begin{enumerate}
\item Многие исследования показывают потенциал, но плохо систематизированы.
\item Инженерная задача по созданию инфраструктуры для коллективной оптимизации не решена до сих пор.
\item Конечным пользователям компиляторов мешает именно отсутствие инструментария, который помог бы им пользоваться обучаемым компилятором так же, как обычным (интерфейс чёрного ящика).
\item В целом это популярная в последнее время тема — Intel запустил Exascale Lab, также IBM и ARM очень заинтересованы и хотят финансировать эти исследования.
\end{enumerate}

\subsection{Цель}
Цель — разработать систему сбора, систематизации, формализации данных о производительности компилируемых программ в зависимости от настроек компилятора, а также выполняющую функции поддержки базы знаний и обучения с целью:
\begin{enumerate}
\item Предсказания производительности программ
\item Оптимизации производительности программ путём изменения настроек компилятора
\end{enumerate}

Пока рассматриваем только 1 критерий --- производительность.

\section{Задачи}
\begin{enumerate}
\item {Выполнить первоначальное наполнение репозитория данными.
Установить окружение и репозиторий локально и внести туда программы из центрального репозитория.
Собрать небольшое количество признаков для начала работы.\\
Предполагаемый объём работы от общего: 5\%.}
\item {Пронаблюдать влияние набора данных, компилятора и архитектуры на производительность. 
Для этого попробовать перебор разных комбинаций этих параметров (в числе которых и настройки компилятора) и сравнить получаемые результаты. Это нужно для набора данных для последующего машинного обучения на них. 
Это не полный перебор, а небольшой просмотр области поиска с целью выяснения реакции программ на изменения для последующей формализации модели производительности.\\
Предполагаемый объём работы от общего: 10\%.}
\item {Формализовать признаки программ, влияние платформы и настроек компилятора на производительность.
Строим модель производительности программы. Это будет делаться изначально вручную с последующими попытками обобщить до автоматического выбора модели с помощью машинного обучения. Пример: начать с линейной регрессии от одного признака (размера входных данных), дальше обнаруживать нестыковки и добавлять новые признаки.
Полезными моделями могут быть вероятностная (она удобна для представления сравнения 2 программ и легко интерпретируется для понимания человеком), деревья решений (легко интерпретируется, относительно быстро строится и работает).
Другие методы, такие как SVM и kNN, могут тоже быть полезны и возможно будут использоваться — но они лишены такой черты, как простота понимания, почему именно такое решение было выбрано моделью. Это затрудняет дальнейшее развитие проекта (поскольку непонятно, в каком месте модели произошла ошибка). Также, применение методов “ad-hoc” не даёт воспроизводимых результатов, на которых можно хорошо строить модель.\\
Предполагаемый объём работы от общего: 35\%.}
\item {Построить предсказывающий модуль
Он может работать в 2 режимах:
	\begin{enumerate}
	\item Предсказание уровня производительности программы при заданных входных параметрах. Полезно в облачных сервисах и параллельных вычислениях (балансировка загрузки). Это регрессионный анализ.
	\item Выбор оптимальных настроек компилятора для получения максимальной производительности при заданной платформе, программе (которой ещё нет в репозитории известных программ) и множестве наборов входных данных. Возможны даже рекомендации по выбору другого компилятора. Это классификация.
	\end{enumerate}
Сюда же попадает первоначальное обучение предсказателя для начала работы системы.\\
Предполагаемый объём работы от общего: 25\%.}
\item {Анализировать проблемы предсказателя (ошибки неправильной классификации или регрессии) с целью улучшить работу.
Это включает в себя как ручной просмотр моделей экспертами с помощью нашего интрументария (без него это делать крайне неудобно) с целью понимания возможной проблемы, так и попытки автоматически улучшить модель с помощью добавления новых признаков или настроек, не учтённых в модели для этого.
Сюда же попадает дообучение при получении новых данных, таких как запуск программы с новыми данными, на новой платформе или с новым компилятором, или данные о новых программах.\\
Предполагаемый объём работы от общего: 20\%}
\item {Улучшить интеграцию в существующие компиляторы, так чтобы для конечного пользователя работа системы происходила прозрачно (интерфейс работы с компилятором не изменялся). Это нужно для вовлечения большего числа разработчиков и экспертов в компьютерных системах для выявления проблем моделей на этапе 5.\\
Предполагаемый объём работы от общего: 5\%.}
\end{enumerate}

Реализация будет использовать Python в качестве основного языка с возможными вызовами scipy и R для реализации сложной обработки данных. Это простой и удобный, популярный и хорошо поддерживаемый язык с большим количеством библиотек. В случае необходимости можно достигнуть высокой производительности с помощью компилятора PyPy.

В качестве базы данных, репозитория пакетов и веб-сервера используется cM — открытый инструментарий на Python.
\section*{Введение}
\addcontentsline{toc}{section}{Введение}%
\subsection*{Цель}
\addcontentsline{toc}{subsection}{Цель}%
Цель данной работы --- произвести анализ производительности программ из набора Polybench с целью построения моделей производительности, а также разработать способы автоматизации построения и улучшения моделей.

\subsection*{Задачи}
\addcontentsline{toc}{subsection}{Задачи}%
Задачи, решаемые в данной работе, перечислены далее в примерном порядке выполнения.
\begin{enumerate}
	\item Построить первоначальную образцовую модель производительности. Эта модель будет простейшей в смысле описываемой зависимости времени выполнения от других свойств программы, набора данных и платформы -- она будет описывать зависимость только от одного свойства. Возможно построение многих таких моделей с целью нахождения подходящей для анализа и расширения. Расширение модели будет производиться для увеличения точности, с которой она описывает поведение реальной программы в реальном окружении.
	\item Выявить недостатки первоначальной модели. Поскольку простейшая модель наверняка не сможет показать практически полезных результатов, необходимо выявить её недостатки и предложить способы их устранения.
	\item Разработать программу анализа модели производительности с целью выявления недостаточности имеющихся в модели признаков. Эта программа должна суметь проанализировать модель с целью определения того, является ли она адекватной -- т.е. хорошо описывающей поведение реальной программы.
	\item Произвести добавление в модель признаков, позволяющих более точно описать производительность программы на данной платформе, в ручном или автоматическом режиме. Это производится для улучшения качества описания поведения моделью производительности. Это будет произведено в ручном режиме с последующей попыткой запрограммировать введение новых признаков. Новые признаки позволят описать более сложную зависимость производительности от свойств исследуюемой системы.
	\item Произвести оценку разработанной программы на наборе моделей, полученных с помощью анализа программ, входящих в состав набора Polybench. Для проверки работы инструментария требуется построить и проанализировать несколько разных моделей. Затем необходимо рассмотреть возникшие проблемы.
\end{enumerate}
\pagebreak
% Теоретическая часть
\part{Теоретическая часть}

\section{Регрессионный анализ}

Hello \cite{lag}.
% Исследовательская часть
\part{Исследовательская часть}
\section{Моделирование производительности программ}
\section{Библиотеки машинного обучения для языка Python}
\subsection{Рассмотрение альтернатив}
\subsubsection{Orange}
\subsubsection{sklearn}
\subsubsection{PyML}
\subsubsection{mlpy}
% Инструментарий коллективной оптимизации
\part{Конструкторская часть}

\section{Реализация модели производительности программ}
% Эксперименты
\section{Экспериментальная часть}

\section{Методология исследования}

Целью исследования является выявление зависимости между характеристиками аппаратного обеспечения и производительностью программы решения СЛАУ. Анализ должен производиться без знания внутренней организации программы (используемого языка программирования, реализованного метода решения и т.д.), поскольку он должен быть автоматизирован и должен учитывать только свойства, присутствующие в модели производительности программы. На данный момент единственная характеристика задачи, присутствующая в модели --- это размер матрицы, задаваемый числом строк и столбцов.

Выдвинем следующую гипотезу: 

на производительность программы будет влиять размер обрабатываемой матрицы, размер кэша процессора и, в меньшей степени, частота процессора. (1)

Объём оперативной памяти не оказывает влияния до тех пор, пока задача целиком помещается в памяти, поэтому это свойство компьютера не учитывается в модели. Данная гипотеза будет подтверждена или опровергнута путём проведения ранжирования признаков (\textit{англ. feature ranking}) \cite{feature-ranking}.

Выдвинем ещё одну гипотезу:

возможно предсказание прозводительности программы на данной аппаратной платформе при обработке задачи данной размерности. (2)

Эта гипотеза будет проверяться проведением регрессионного анализа и последующего предсказания регрессионной моделью ожидаемой производительности. При этом будет применена методология перекрёстной проверки, как описано в разделе \ref{cross-validation}.

Ввиду ограниченных временных и материальных ресурсов, ограничимся запуском программы на трёх процессорах фирмы Intel:  Core 2 Quad Q8200 с частотой 2,33 ГГц и кэшем третьего уровня объёма 2 МБ, Xeon E5430 с частотой 2,66 ГГц и кэшем третьего уровня объёма 6 МБ, и Core i5 M 460 с частотой 2,53 ГГц и кэшем третьего уровня объёма 3 МБ. Эти процессоры принадлежат к разным семействам (для настольных компьютеров, для серверов и для ноутбуков, соответственно) и выпущены в разное время (2007, 2008, 2010 годы соответственно).

Произведём достаточно много экспериментов на каждом компьютере --- по 1000 запусков. Каждый запуск производится с случайно выбранным размером матрицы. Случайные величины числа столбцов и числа строк матрицы равномерно распределены в диапазоне [2:1024], следовательно, размер матрицы колеблется от $2 \times 2$ до $1024 \times 1024$.

Данный сценарий исследования называется случайным перебором размера матрицы. Он реализован в описываемой системе проведения исследований, как указано в разделе \ref{random-exploration}.

Измерение времени исполнения исследуемой программы производится с использованием процедуры калибровки, как описано в записке \cite{adaptor}. Принципиально, процедура заключается в многократном запуске программы с увеличением числа запусков до тех пор, пока точность измерения времени одного запуска, производимое путём деления общего времени исполнения на число запусков, не достигнет необходимой величины.

% Дальнейшая работа
\section{Заключение}
В рамках работы были решены задачи, перечисленные ниже.
\begin{itemize}
	\item Построена первоначальная образцовая модель производительности.
	\item Выявлены недостатки первоначальной модели.
	\item Разработана программа анализа модели производительности с целью выявления недостаточности имеющихся в модели признаков.
	\item Произведено добавление в модель признаков, позволяющих более точно описать производительность программы на данной платформе, в ручном и автоматическом режиме.
	\item Произведена оценка разработанной программы на наборе моделей, полученных с помощью анализа программ, входящих в состав набора Polybench.
\end{enumerate}
% Название библиографии
\renewcommand\bibname{Список использованных источников}
% Список использованных источников
\bibliography{biblio/course}
\end{document}
