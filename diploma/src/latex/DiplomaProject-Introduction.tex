\chapter{Введение}
Разработка современного оптимизирующего промышленного компилятора --- сложная задача, требующая огромных вложений человеческих ресурсов. Больш\'{a}я часть этой сложности обусловлена необходимостью тонкой настройки эвристических методов оптимизации исполняемого кода, производимой вручную путём просмотра генерируемого кода и дампов промежуточного представления структур компилятора. Более того, сами правила обнаружения случаев, в которых можно оптимизировать код, создаются человеком, что также трудоёмко и является далеко не исчерпывающим способом увеличения производительности результирующего кода. Принципиально задача подбора самих эвристик и их настройки может решаться автоматически, как уже доказали работы {}.