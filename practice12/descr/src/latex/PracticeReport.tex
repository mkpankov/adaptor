% Размер бумаги А4, шрифт 12 пунктов
\documentclass[a4paper,12pt]{report}
% Перекодировка шрифтов
\usepackage[T2A]{fontenc}
% Кодировка: utf8
\usepackage[utf8]{inputenc}
% Используем русский и английский языки с переносами
\usepackage[english,russian]{babel}
% Подключаем нужные пакеты расширений
\usepackage{amssymb,amsfonts,amsmath,mathtext,cite,enumerate,float}

% Вставка рисунков
\usepackage{graphicx}
% Путь к рисункам
\graphicspath{{../pictures/}}

% Рыба
\usepackage{lipsum}
% Подсчёт объектов
\usepackage{totcount}
\regtotcounter{page}
\regtotcounter{figure}
\regtotcounter{table}

% Изменение размеров шрифтов
\usepackage{scrextend}
% Размер по умолчанию — 14 пунктов, интервал полуторный (14*1.25)
\changefontsizes[17.5pt]{14pt}
% Отступ первой строки параграфа после начала раздела
\usepackage{indentfirst}
% Отступ начала параграфа
\setlength{\parindent}{1cm}

% Заменяем библиографию с квадратных скобок на точку
\makeatletter
\renewcommand{\@biblabel}[1]{#1.}
\makeatother

% Геометрия страницы
\usepackage{geometry}
% Поля
\geometry{left=3cm}
\geometry{right=1cm}
\geometry{top=2cm}
\geometry{bottom=2cm}

% Меняем везде перечисления на цифра.цифра
\renewcommand{\theenumi}{\arabic{enumi}.}
\renewcommand{\labelenumi}{\arabic{enumi}.}
\renewcommand{\theenumii}{\arabic{enumii}.}
\renewcommand{\labelenumii}{\arabic{enumi}.\arabic{enumii}.}
\renewcommand{\theenumiii}{\arabic{enumiii}.}
\renewcommand{\labelenumiii}{\arabic{enumi}.\arabic{enumii}.\arabic{enumiii}.}

% Изменяем \part и \chapter из report.cls чтобы они не выводили нумерованные заголовки
% Рецепт отсюда: http://tex.stackexchange.com/questions/70621/part-and-chapter-commands-output-both-numbered-and-named-headings/70624
\makeatletter
    \def\@part[#1]#2{%
        \ifnum \c@secnumdepth >-2\relax
        \refstepcounter{part}%
        \addcontentsline{toc}{part}{#1}%
        \else
        \addcontentsline{toc}{part}{#1}%
        \fi
        \markboth{}{}%
        {
            \centering
            \interlinepenalty \@M
            \normalfont
            \Huge \bfseries #2\par}%
        \@endpart}
    \def\@chapter[#1]#2{\ifnum \c@secnumdepth >\m@ne
        \refstepcounter{chapter}%
        \typeout{\@chapapp\space\thechapter.}%
        \addcontentsline{toc}{chapter}%
        {#1}%
        \else
        \addcontentsline{toc}{chapter}{#1}%
        \fi
        \chaptermark{#1}%
        \addtocontents{lof}{\protect\addvspace{10\p@}}%
        \addtocontents{lot}{\protect\addvspace{10\p@}}%
        \if@twocolumn
        \@topnewpage[\@makechapterhead{#2}]%
        \else
        \@makechapterhead{#2}%
        \@afterheading
        \fi}
    \def\@makechapterhead#1{%
        \vspace*{50\p@}%
        {\parindent \z@ \raggedright \normalfont
            \interlinepenalty\@M
            \Huge \bfseries #1\par\nobreak
            \vskip 40\p@
    }}
    \renewcommand{\thesection}{\arabic{section}} % Удаляем \thechapter из \thesection
\makeatother

% Для веб-адресов в библиографии
\usepackage{url}
% Библиография в порядке упоминания в тексте
\bibliographystyle{unsrt}

\begin{document}
% Титульный лист
\input{PracticeReport-Title}
% Введение
\section*{Введение}
\addcontentsline{toc}{section}{Введение}%
\subsection*{Цель}
\addcontentsline{toc}{subsection}{Цель}%
Цель данной работы --- произвести анализ производительности программ из набора Polybench с целью построения моделей производительности, а также разработать способы автоматизации построения и улучшения моделей.

\subsection*{Задачи}
\addcontentsline{toc}{subsection}{Задачи}%
Задачи, решаемые в данной работе, перечислены далее в примерном порядке выполнения.
\begin{enumerate}
	\item Построить первоначальную образцовую модель производительности. Эта модель будет простейшей в смысле описываемой зависимости времени выполнения от других свойств программы, набора данных и платформы -- она будет описывать зависимость только от одного свойства. Возможно построение многих таких моделей с целью нахождения подходящей для анализа и расширения. Расширение модели будет производиться для увеличения точности, с которой она описывает поведение реальной программы в реальном окружении.
	\item Выявить недостатки первоначальной модели. Поскольку простейшая модель наверняка не сможет показать практически полезных результатов, необходимо выявить её недостатки и предложить способы их устранения.
	\item Разработать программу анализа модели производительности с целью выявления недостаточности имеющихся в модели признаков. Эта программа должна суметь проанализировать модель с целью определения того, является ли она адекватной -- т.е. хорошо описывающей поведение реальной программы.
	\item Произвести добавление в модель признаков, позволяющих более точно описать производительность программы на данной платформе, в ручном или автоматическом режиме. Это производится для улучшения качества описания поведения моделью производительности. Это будет произведено в ручном режиме с последующей попыткой запрограммировать введение новых признаков. Новые признаки позволят описать более сложную зависимость производительности от свойств исследуюемой системы.
	\item Произвести оценку разработанной программы на наборе моделей, полученных с помощью анализа программ, входящих в состав набора Polybench. Для проверки работы инструментария требуется построить и проанализировать несколько разных моделей. Затем необходимо рассмотреть возникшие проблемы.
\end{enumerate}
% Инструментарий коллективной оптимизации
\part{Конструкторская часть}

\section{Реализация модели производительности программ}
% Название библиографии
\renewcommand\bibname{Список использованных источников}
% Список использованных источников
\bibliography{biblio/ref}
\end{document}