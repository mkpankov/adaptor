\chapter{Проверка оптимизаций, сохраняющих структуру программы}

Опишем стратегию \emph{voc} и теорию валидации оптимизаций.

Компилятор получает на вход \emph{исходную программу}, написанную на высокоуровневом языке, преобразует её в \emph{промежуточное представление} и затем применяет набор оптимизаций к программе --- начиная с классических глобальных оптимизаций, независящих от архитектуры, и заканчивая архитектурно-зависимыми, такими, как распределение регистров и планирование инструкций. Обычно эти оптимизации производятся в несколько проходов (до 15 в некоторых компиляторах), и каждый проход применяет оптимизацию определённого типа.
