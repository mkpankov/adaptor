\documentclass{disser}

\usepackage[english,russian]{babel}
\usepackage[utf8x]{inputenc}

\begin{document}

\begin{abstract}
Статья рассматривает статистический метод моделирования быстродействия программ на компьютерах общего назначения ,,Velocitas''. Дано описание метода. Произведена программная реализация инструментария моделирования быстродействия программ ,,Adaptor'' включающего в себя данный метод. Произведена оценка эффективности метода ,,Velocitas''. Достигнута более высокая по сравнению с аналогами точность предсказания быстродействия.
\end{abstract}

\section*{Введение}
\subsection*{Обзор области}
В ходе проектирования компьютерных систем часто возникает задача оценки быстродействия определённых программ на данной системе. Традиционным методом решения данной задачи является эмуляция исполнения программы \cite{emulation}. У этого метода есть несколько недостатков. Во-первых, он требует полной реализации поведения эмулируемого компьютера, что требует больших вложений ресурсов \cite{emulation-complexity}. Во-вторых, скорость выполнения программы на эмулируемом компьютере меньше скорости реального выполнения в сотни и тысячи раз \cite{emulation-speed}.
\subsection*{Актуальность}
Как уже было отмечено выше, одной из областей применения моделирования компьютерных систем является оценка быстродействия ещё не произведённого компьютера на этапе его разработки.

Другой пример "--- оценка быстродействия программы при итеративной компиляции.
\section{Методология}
\subsection{Метод моделирования быстродействия программ "Velocitas"}

\subsection{Метод измерения времени исполнения исследуемой программы}

\section{Программная реализация}
\subsection{Архитектура. Общая информация}

\subsection{Подсистемы сервера}

\subsection{Подсистемы клиента}

\section{Исследование эффективности}
\subsection{Методология и экспериментальное окружение}

\subsection{Анализ результатов}

\section*{Заключение}

\end{document}