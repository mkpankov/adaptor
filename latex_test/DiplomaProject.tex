\documentclass[a4paper,12pt]{report} %размер бумаги устанавливаем А4, шрифт 12пунктов
\usepackage[T2A]{fontenc}
\usepackage[utf8]{inputenc}%включаем свою кодировку: koi8-r или utf8 в UNIX, cp1251 в Windows
\usepackage[english,russian]{babel}%используем русский и английский языки с переносами
\usepackage{amssymb,amsfonts,amsmath,mathtext,cite,enumerate,float} %подключаем нужные пакеты расширений
\usepackage{subfigure}
\usepackage{wrapfig}
\usepackage[pdftex]{graphicx} %хотим вставлять в диплом рисунки?
\graphicspath{{images/}}%путь к рисункам

\makeatletter
\renewcommand{\thefigure}{\thechapter.\arabic{figure}}
\renewcommand{\@biblabel}[1]{#1.} % Заменяем библиографию с квадратных скобок на точку:
\makeatother

\usepackage{geometry} % Меняем поля страницы
\geometry{left=2cm}% левое поле
\geometry{right=1.5cm}% правое поле
\geometry{top=1cm}% верхнее поле
\geometry{bottom=2cm}% нижнее поле

\renewcommand{\theenumi}{\arabic{enumi}}% Меняем везде перечисления на цифра.цифра
\renewcommand{\labelenumi}{\arabic{enumi}}% Меняем везде перечисления на цифра.цифра
\renewcommand{\theenumii}{.\arabic{enumii}}% Меняем везде перечисления на цифра.цифра
\renewcommand{\labelenumii}{\arabic{enumi}.\arabic{enumii}.}% Меняем везде перечисления на цифра.цифра
\renewcommand{\theenumiii}{.\arabic{enumiii}}% Меняем везде перечисления на цифра.цифра
\renewcommand{\labelenumiii}{\arabic{enumi}.\arabic{enumii}.\arabic{enumiii}.}% Меняем везде перечисления на цифра.цифра

\begin{document}
\begin{titlepage}
    \newgeometry{top=2cm,bottom=2cm,left=2cm,right=2cm}
    \newpage
    
    \begin{center}
        Федеральное агенство по образованию Российской Федерации \\
        Московский Государственный Технический Университет \\*
        имени Н.Э.Баумана \\*
        \vspace{-12mm}
        \begin{figure}[h]
            \center{\includegraphics[width=0.2\linewidth]{symbol}}
        \end{figure}
        \vspace{-16mm}
        \hrulefill
    \end{center}
    \center{Факультет «Робототехника и комплексная автоматизация»\\
    		Кафедра «Системы Автоматизированного Проектирования»}
    \begin{center}
        \Large Пояснительная записка \\ к дипломному проекту на тему:
    \end{center}
    
    \vspace{2.5em}
    
    \begin{center}
        \textsc{\textbf{Инструментарий коллективной оптимизации программ.}}
    \end{center}
    
    \vspace{6em}
    
    \begin{flushleft}
        \hspace{6.5cm}Студент--дипломник \hrulefill Панков М.К. \\
        \vspace{1.5em}
        \hspace{6.5cm}Научный руководитель \hrulefill Карпенко А.П.\\
        \vspace{1.5em}
        \hspace{6.5cm}Рецензент \hrulefill ХХХ\\
        \vspace{1.5em}
        \hspace{6.5cm}Зав. кафедрой РК-6 \hrulefill Карпенко А.П.
    \end{flushleft}
    
    \vspace{\fill}
    
    \begin{center}
        Москва, \\*
        2013
    \end{center}

\end{titlepage}% это титульный лист
\tableofcontents % это оглавление, которое генерируется автоматически
\begin{figure}[h]
\center{\includegraphics[width=1\linewidth]{Penguins}}
\caption{Тестовая тупая картинка.}
\label{img:image}
\end{figure}
\begin{figure}[h]
\begin{minipage}[h]{0.49\linewidth}
\center{\includegraphics[width=0.5\linewidth]{Lighthouse} \\ а)}
\end{minipage}
\hfill
\begin{minipage}[h]{0.49\linewidth}
\center{\includegraphics[width=0.5\linewidth]{Penguins} \\ б)}
\end{minipage}
\caption{Замок а) и Пингвины б).}
\label{img:castle_penguins}
\end{figure}
Как вы можете видеть, вот тут у нас замок \ref{img:castle_penguins}, а тут — пингвины \ref{img:castle_penguins}.
\begin{figure}[h]
\begin{center}
\begin{minipage}[h]{0.4\linewidth}
\includegraphics[width=1\linewidth]{Lighthouse}
\caption{Исходное изображение.} %% подпись к рисунку
\label{img:lighthouse} %% метка рисунка для ссылки на него
\end{minipage}
\hfill 
\begin{minipage}[h]{0.4\linewidth}
\includegraphics[width=1\linewidth]{Penguins}
\caption{Закодированное изображение.}
\label{img:penguins}
\end{minipage}
\end{center}
\end{figure}
Вот тут у нас замок на рис. \ref{img:lighthouse} и пингвины на рис. \ref{img:penguins}.
\begin{figure}[ht!]
\vspace{-4ex}
\centering
\subfigure[]{
\includegraphics[width=0.25\linewidth]{Lighthouse}
\label{img:lig}
}
\hspace{4ex}
\subfigure[]{
\includegraphics[width=0.25\linewidth]{Penguins}
\label{img:pen1}
}
\hspace{4ex}
\subfigure[]{
\includegraphics[width=0.24\linewidth]{Penguins}
\label{img:pen2}
}
\caption{Coupling cases for the DM models: 
\subref{img:lig} Lighthouse
\subref{img:pen1} Penguins 1 ; 
\subref{img:pen2}} Penguins 2.
\label{fig:threeDMcases}
\end{figure}
Замок \ref{img:lig}, пингвины 1 \ref{img:pen1}, пингвины 2 \ref{img:pen2}.
\begin{figure}[H]
\begin{minipage}[h]{0.32\linewidth}
\center{\includegraphics[width=0.8\linewidth]{image}}
\end{minipage}
\hfill
\begin{minipage}[h]{0.32\linewidth}
\center{\includegraphics[width=0.4\linewidth]{image}}
\end{minipage}
\hfill
\begin{minipage}[h]{0.32\linewidth}
\center{\includegraphics[width=0.9\linewidth]{image}}
\end{minipage}
\begin{minipage}[h]{1\linewidth}
\begin{tabular}{p{0.32\linewidth}p{0.32\linewidth}p{0.32\linewidth}}
\centering а) & \centering б) & \centering в) \\
\end{tabular}
\end{minipage}
\vspace*{-1cm}
\caption{Процесс кодирования: а) оригинальное изображение, б) ФРТ киноформа, в) закодированное изображение.}
\label{ris:correlationsignals}
\end{figure}
\begin{minipage}[h]{1\linewidth}
\begin{tabular}{p{0.32\linewidth}p{0.32\linewidth}p{0.32\linewidth}}
\centering а) & \centering б) & \centering в) \\
\end{tabular}
\end{minipage}
Текст бла-бла-бла.
\begin{wrapfigure}[16]{r}{0.5\linewidth} 
\vspace{-5ex}
\includegraphics[width=\linewidth]{image}
\caption{Some caption}
\label{fig:somelabel}
\end{wrapfigure}
\end{document}