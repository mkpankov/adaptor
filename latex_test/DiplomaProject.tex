\documentclass[a4paper,12pt]{report} %размер бумаги устанавливаем А4, шрифт 12пунктов
\usepackage[T2A]{fontenc}
\usepackage[utf8]{inputenc}%включаем свою кодировку: koi8-r или utf8 в UNIX, cp1251 в Windows
\usepackage[english,russian]{babel}%используем русский и английский языки с переносами

\addto\captionsrussian{% Replace "english" with the language you use
    \renewcommand{\contentsname}%
    {Whatever}%
}

%\makeatletter %%%%% <---- Starting chapter without a pagebreak
%    \renewcommand\chapter{\par%
%    \thispagestyle{plain}%
%    \global\@topnum\z@
%    \@afterindentfalse \secdef\@chapter\@schapter} 
%\makeatother %%%%% <---- Starting chapter without a pagebreak

%\makeatletter % эта строка НЕОБХОДИМА!
%\renewcommand{\@chapapp}{Глава} % необязательная строчка.
%% определяет значение \@chapapp (используется ниже)
%\renewcommand{\@makechapterhead}[1]{% Начало макроопределения
%    \vspace*{50 pt}% Пустое место вверху страницы
%    {\parindent=0pt
%        \raggedright \normalfont\huge\bfseries
%        \@chapapp{} % \@chapapp печатает слово "Glava" (см. выше)
%        % вот эту строку
%        % \thechapter \par % номер главы - в отдельной строке
%        % и ещё вот эту я убрал, и у меня наступило ШЧАСТЬЕ
%        % \thechapter -- печать счётчика глав; \par -- насильственно
%        % закончить абзац
%        \vspace{20 pt} % между словом "Glava" и ее заголовком
%        % и ещё эту строчку я грохнул, ибо нефиг
%        % \normalfont\Huge\bfseries #1\par % заголовок главы
%        \nopagebreak % чтоб не оторвать заголовок от текста
%        \vspace{40 pt} % между заголовком и текстом
%    }% конец группы.
%}% конец макроопределения
%\makeatother % эта строка НЕОБХОДИМА!

% Modified \part and \chapter from report.cls
\makeatletter
    \def\@part[#1]#2{%
        \ifnum \c@secnumdepth >-2\relax
        \refstepcounter{part}%
        \addcontentsline{toc}{part}{#1}%
        \else
        \addcontentsline{toc}{part}{#1}%
        \fi
        \markboth{}{}%
        {
            \centering
            \interlinepenalty \@M
            \normalfont
            \Huge \bfseries #2\par}%
        \@endpart}
    \def\@chapter[#1]#2{\ifnum \c@secnumdepth >\m@ne
        \refstepcounter{chapter}%
        \typeout{\@chapapp\space\thechapter.}%
        \addcontentsline{toc}{chapter}%
        {#1}%
        \else
        \addcontentsline{toc}{chapter}{#1}%
        \fi
        \chaptermark{#1}%
        \addtocontents{lof}{\protect\addvspace{10\p@}}%
        \addtocontents{lot}{\protect\addvspace{10\p@}}%
        \if@twocolumn
        \@topnewpage[\@makechapterhead{#2}]%
        \else
        \@makechapterhead{#2}%
        \@afterheading
        \fi}\makeatother
    \def\@makechapterhead#1{%
        \vspace*{50\p@}%
        {\parindent \z@ \raggedright \normalfont
            \interlinepenalty\@M
            \Huge \bfseries #1\par\nobreak
            \vskip 40\p@
    }}
    \renewcommand{\thesection}{\arabic{section}}% Remove \thechapter from \thesection}
\makeatother

\usepackage{amssymb,amsfonts,amsmath,mathtext,cite,enumerate,float} %подключаем нужные пакеты расширений
\usepackage{subfigure}
\usepackage{wrapfig}
\usepackage{longtable}
\usepackage{setspace} %для установки интервалов в стандартных элементах
\usepackage{tocvsec2}
\usepackage{graphicx} %хотим вставлять в диплом рисунки?
\graphicspath{{images/}}%путь к рисункам

\makeatletter
\renewcommand{\thefigure}{\thechapter.\arabic{figure}}
\renewcommand{\@biblabel}[1]{#1.} % Заменяем библиографию с квадратных скобок на точку:
\makeatother

\usepackage{geometry} % Меняем поля страницы
\geometry{left=2cm}% левое поле
\geometry{right=1.5cm}% правое поле
\geometry{top=1cm}% верхнее поле
\geometry{bottom=2cm}% нижнее поле

\renewcommand{\theenumi}{\arabic{enumi}}% Меняем везде перечисления на цифра.цифра
\renewcommand{\labelenumi}{\arabic{enumi}}% Меняем везде перечисления на цифра.цифра
\renewcommand{\theenumii}{.\arabic{enumii}}% Меняем везде перечисления на цифра.цифра
\renewcommand{\labelenumii}{\arabic{enumi}.\arabic{enumii}.}% Меняем везде перечисления на цифра.цифра
\renewcommand{\theenumiii}{.\arabic{enumiii}}% Меняем везде перечисления на цифра.цифра
\renewcommand{\labelenumiii}{\arabic{enumi}.\arabic{enumii}.\arabic{enumiii}.}% Меняем везде перечисления на цифра.цифра
\onehalfspacing % Полуторный интервал между строками
\setcounter{tocdepth}{3} %%where n is the level, starting with 0 (chapters only)
\begin{document}
    % \settocdepth{section} %% установка глубины оглавления
    \begin{titlepage}
    \newgeometry{top=2cm,bottom=2cm,left=2cm,right=2cm}
    \newpage
    
    \begin{center}
        Федеральное агенство по образованию Российской Федерации \\
        Московский Государственный Технический Университет \\*
        имени Н.Э.Баумана \\*
        \vspace{-12mm}
        \begin{figure}[h]
            \center{\includegraphics[width=0.2\linewidth]{symbol}}
        \end{figure}
        \vspace{-16mm}
        \hrulefill
    \end{center}
    \center{Факультет «Робототехника и комплексная автоматизация»\\
    		Кафедра «Системы Автоматизированного Проектирования»}
    \begin{center}
        \Large Пояснительная записка \\ к дипломному проекту на тему:
    \end{center}
    
    \vspace{2.5em}
    
    \begin{center}
        \textsc{\textbf{Инструментарий коллективной оптимизации программ.}}
    \end{center}
    
    \vspace{6em}
    
    \begin{flushleft}
        \hspace{6.5cm}Студент--дипломник \hrulefill Панков М.К. \\
        \vspace{1.5em}
        \hspace{6.5cm}Научный руководитель \hrulefill Карпенко А.П.\\
        \vspace{1.5em}
        \hspace{6.5cm}Рецензент \hrulefill ХХХ\\
        \vspace{1.5em}
        \hspace{6.5cm}Зав. кафедрой РК-6 \hrulefill Карпенко А.П.
    \end{flushleft}
    
    \vspace{\fill}
    
    \begin{center}
        Москва, \\*
        2013
    \end{center}

\end{titlepage}% это титульный лист
    \begin{spacing}{0.9}
        \tableofcontents % это оглавление, которое генерируется автоматически
    \end{spacing}
    \part{Часть Первая}
    \chapter{Глава Раз}
    \section{Раздел Один}
    \subsection{Подраздел Х}
    \begin{figure}[h]
        \center{\includegraphics[width=1\linewidth]{Penguins}}
        \caption{Тестовая тупая картинка.}
        \label{img:image}
    \end{figure}
    \subsection{Замки с Пингвинами}
    \begin{figure}[h]
        \begin{minipage}[h]{0.49\linewidth}
            \center{\includegraphics[width=0.5\linewidth]{Lighthouse} \\ а)}
        \end{minipage}
        \hfill
        \begin{minipage}[h]{0.49\linewidth}
            \center{\includegraphics[width=0.5\linewidth]{Penguins} \\ б)}
        \end{minipage}
        \caption{Замок а) и Пингвины б).}
        \label{img:castle_penguins}
    \end{figure}
    Как вы можете видеть, вот тут у нас замок \ref{img:castle_penguins}, а тут — пингвины \ref{img:castle_penguins}.
    \begin{figure}[h]
        \begin{center}
            \begin{minipage}[h]{0.4\linewidth}
                \includegraphics[width=1\linewidth]{Lighthouse}
                \caption{Исходное изображение.} %% подпись к рисунку
                \label{img:lighthouse} %% метка рисунка для ссылки на него
            \end{minipage}
            \hfill 
            \begin{minipage}[h]{0.4\linewidth}
                \includegraphics[width=1\linewidth]{Penguins}
                \caption{Закодированное изображение.}
                \label{img:penguins}
            \end{minipage}
        \end{center}
    \end{figure}
    Вот тут у нас замок на рис. \ref{img:lighthouse} и пингвины на рис. \ref{img:penguins}.
    \begin{figure}[ht!]
        \vspace{-4ex}
        \centering
        \subfigure[]{
            \includegraphics[width=0.25\linewidth]{Lighthouse}
            \label{img:lig}
        }
        \hspace{4ex}
        \subfigure[]{
            \includegraphics[width=0.25\linewidth]{Penguins}
            \label{img:pen1}
        }
        \hspace{4ex}
        \subfigure[]{
            \includegraphics[width=0.24\linewidth]{Penguins}
            \label{img:pen2}
        }
        \caption{Coupling cases for the DM models: 
        \subref{img:lig} Lighthouse
        \subref{img:pen1} Penguins 1 ; 
        \subref{img:pen2}} Penguins 2.
        \label{fig:threeDMcases}
    \end{figure}
    Замок \ref{img:lig}, пингвины 1 \ref{img:pen1}, пингвины 2 \ref{img:pen2}.
    \begin{figure}[H]
        \begin{minipage}[h]{0.32\linewidth}
            \center{\includegraphics[width=0.8\linewidth]{image}}
        \end{minipage}
        \hfill
        \begin{minipage}[h]{0.32\linewidth}
            \center{\includegraphics[width=0.4\linewidth]{image}}
        \end{minipage}
        \hfill
        \begin{minipage}[h]{0.32\linewidth}
            \center{\includegraphics[width=0.9\linewidth]{image}}
        \end{minipage}
        \begin{minipage}[h]{1\linewidth}
            \begin{tabular}{p{0.32\linewidth}p{0.32\linewidth}p{0.32\linewidth}}
                \centering а) & \centering б) & \centering в) \\
            \end{tabular}
        \end{minipage}
        \vspace*{-1cm}
        \caption{Процесс кодирования: а) оригинальное изображение, б) ФРТ киноформа, в) закодированное изображение.}
        \label{ris:correlationsignals}
    \end{figure}
    \begin{minipage}[h]{1\linewidth}
        \begin{tabular}{p{0.32\linewidth}p{0.32\linewidth}p{0.32\linewidth}}
            \centering а) & \centering б) & \centering в) \\
        \end{tabular}
    \end{minipage}
    Текст бла-бла-бла.
    \begin{wrapfigure}[16]{r}{0.5\linewidth} 
        \vspace{-5ex}
        \includegraphics[width=\linewidth]{image}
        \caption{Some caption}
        \label{fig:somelabel}
    \end{wrapfigure}
    \pagebreak
    \begin{center}
        \begin{tabular}{ccc}
            Расширение краёв: & \textbf{1,0-1,4} & размер ФРТ \\
            Аподизация: & \textbf{0,25-0,30}& размер ФРТ \\
            Сглаживания краёв: & \textbf{0,25-0,50}& размер ФРТ \\
        \end{tabular}
        \begin{table}[H]
            \caption{Исправьте это на подпись к таблице}
            \label{tabular:timesandtenses}
            \begin{center}
                \begin{tabular}{ccc}
                Расширение краёв: & \textbf{1,0-1,4} & размер ФРТ \\
                Аподизация: & \textbf{0,25-0,30}& размер ФРТ \\
                Сглаживания краёв: & \textbf{0,25-0,50}& размер ФРТ \\
                \end{tabular}
            \end{center}
        \end{table}
    \end{center}
    \begin{table}[H]
        \caption{\label{tab:canonsummary}Измерительные характеристики цифровой камеры Canon EOS 400D.}
        \begin{center}
            \begin{tabular}{|c|c|}
                \hline
                Параметр & Значение \\
                \hline
                Разрешение & $3888 \times 2592$ \\
                Размер сенсора & $22.2 \times 14.8$ мм \\
                АЦП & 12~bit\\
                \hline
                \multicolumn{2}{|c|}{Результаты измерений} \\
                \hline
                Темновое смещение (BLO) & 256 \\
                Максимальный линейный сигнал & 3070~DN \\
                Значение насыщения & 3470~DN \\
                \hline
            \end{tabular}
        \end{center}
    \end{table}
    \begin{longtable}[h]{lp{0.7\linewidth}}
        $A$ & area of the (geometrical) pixel [m\textsuperscript2] \\
        $c$ & Speed of light $\approx 3 \cdot 10^8$ m/s \\
        $DYN_{in}$ & Input dynamic range [1] \\
        $DYN_{out}$ & Output dynamic range~[1] \\
        $E$ & irradiance on the sensor surface~[W/m\textsuperscript2] \\
        $F$ & Non-whiteness coefficient \\ 
        $h$ & Planck's constant h $\approx6.63 \cdot 10^{-34} Js$ \\
        $K$ & overall system gain~[DN/e-] \\
        $k_d$ & Doubling temperature of the dark current~[$^\circ$C] \\
        $N_d$ & dark current~[e-/s] \\
        $N_{d30}$ & dark current for a housing temperature of $30^\circ C$~[e-/s]\textsuperscript2 \\
        $S_g^2$ & variance coefficient of the spatial gain noise~[\%] \\
    \end{longtable}
    \begin{table}[H]
    \caption{\label{tab:bolts} Нестандартные болты для левой резьбы.}
    \begin{center}
    \begin{tabular}{|c|c|c|}
    \hline
    & \multicolumn{2}{c|}{Диаметр} \\
    \cline{2-3}
    \raisebox{1.5ex}[0cm][0cm]{Нестандартные болты}
    & Норма & Разброс \\
    \hline
    Размеры & 10 мм & 1 мм \\
    \hline
    \end{tabular}
    \end{center}
    \end{table}
    \rotatebox{90}{%это обеспечивает поворот любого объекта
        \begin{minipage}{1.5\linewidth}
            \begin{table}[H]
                \caption{Сравнение количественных данных по влиянию искажений на результат декодирования}
                \begin{center}
                    \begin{tabular}{|p{5cm}c|c|}
                        \hline
                        \multicolumn{3}{|c|}{\textbf{Графические изображения}}\\
                        \hline
                        \textbf{Характеристика} & \textbf{8-битные} & \textbf{16-битные}\\
                        \hline
                        \multicolumn{3}{|c|}{Влияние шумов квантования}\\
                        \hline
                        Шумы квантования
                        & $6.3 \div 22.6\%$
                        & $0.2 \div 0.5\%$ \\
                        \hline
                        \multicolumn{3}{|c|}{Влияние шумов фотоприёмника}\\
                        \hline
                        Шумы фотоприёмника
                        & $8.8 \div 87.5\%$
                        & $7.1 \div 87.6\%$ \\
                        \hline
                        \multicolumn{3}{|c|}{Влияние особенностей ФРТ}\\
                        \hline
                        декодирование ФРТ той же реализацией шума
                        & $6 \div 13\%$
                        & $0.5 \div 15\%$ \\
                        \hline
                        декодирование ФРТ с другой реализацией шума
                        & $6 \div 14\%$
                        & $0.3 \div 12\%$ \\
                        \hline
                        декодирование идеальным ФРТ
                        & $6 \div 6.5\%$
                        & $0.2 \div 6.4\%$ \\
                        \hline
                    \end{tabular}
                \end{center}
            \end{table}
        \end{minipage}
    }
    \pagebreak
    \begin{center}
        \begin{minipage}{1\textwidth}
            Проверка ололо пыщ-пыщ \textbf{полужирный}, \textit{курсив}, \underline{подчёркнутый}. \textsc{Капитель вот тут ололо}. \\
            Текст ``в кавычках'', ``в кавычках ё``, <<в кавычках>>, ,,в кавычках``.
            Обратный слэш вротмненоги \textbackslash. \\
            Тильда. \textasciitilde. \\
            20\% \No26.\\
            Б\'{о}льшие шумы. Больш\'{и}е шумы.\\
            $ \alpha \beta \gamma $
        \end{minipage}
    \end{center}
    
    \begin{enumerate}
        \item это первое,
        \item а это второе,
        \item и последнее.
    \end{enumerate}
    
    \chapter*{Библиография}
    \addcontentsline{toc}{chapter}{Библиография}
    
%    \addtocounter{counter}{value} увеличивает указанный счётчик counter на значение value.
%    
%    \newcounter{newcounter}[oldcounter] создаёт новый счётчик newcounter; опция нужна для связывания старого счётчика oldcounter и нового newcounter.
%    
%    \setcounter{counter}{value} устанавливает счётчик counter в заданное значение value.
%    
%    \value{counter} выдаёт значение счётчика counter и вообще полезная команда для разного рода арифметических операций со значениями счётчиков.
%    
%    \renewcommand{cmd}[args][opt]{def} общая команда для создания или переопределения новых команд.
\end{document}